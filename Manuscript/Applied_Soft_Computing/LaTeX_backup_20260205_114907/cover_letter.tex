\documentclass[11pt]{letter}
\usepackage[margin=1in]{geometry}
\usepackage{hyperref}

\signature{ZhiHan Wang\\
Corresponding Author\\
SClab, China University of Petroleum (Beijing)\\
Email: wangzhihan@cup.edu.cn}

\address{ZhiHan Wang\\
SClab\\
China University of Petroleum (Beijing)\\
Beijing 102249, China\\
Email: wangzhihan@cup.edu.cn}

\begin{document}

\begin{letter}{Editor-in-Chief\\
Applied Soft Computing\\
Elsevier}

\opening{Dear Editor,}

We submit our manuscript titled \textbf{"Deep Reinforcement Learning for Vertical Layered Queueing Systems in Urban Air Mobility: A Comparative Study of 15 Algorithms"} for consideration in \textit{Applied Soft Computing}.

\textbf{Research Contribution:}

This work presents the first comprehensive evaluation of deep reinforcement learning algorithms for vertical layered queueing systems in Urban Air Mobility contexts. Through systematic experimentation with 15 state-of-the-art algorithms across 500,000 timesteps and rigorous statistical validation, we establish six principal contributions:

\begin{enumerate}
\item \textbf{Algorithm Performance:} We demonstrate that DRL algorithms achieve 59.9\% performance improvement over traditional heuristic methods ($p < 0.001$), with A2C emerging as the optimal choice for production deployment (4,437.86 reward, fastest convergence).

\item \textbf{Structural Design Principles:} We identify and validate the capacity-flow matching principle, showing that inverted pyramid configurations [8,6,4,3,2] consistently outperform normal pyramid structures by 9.7\%-19.7\% across load conditions, providing direct design guidelines for UAM infrastructure planning.

\item \textbf{Capacity Paradox Discovery:} We discover and validate a counter-intuitive capacity paradox where low-capacity systems (K=10) outperform high-capacity systems (K=30+) by orders of magnitude under extreme load conditions ($\geq$8$\times$ baseline), challenging conventional capacity planning assumptions.

\item \textbf{Architectural Design Validation:} Through ablation studies, we demonstrate that capacity-aware action clipping is essential for achieving optimal performance, with 66\% performance degradation when this constraint is removed despite identical network capacity (821K parameters).

\item \textbf{Performance-Stability Trade-off:} Through comprehensive ablation studies comparing HCA2C with baseline algorithms (A2C, PPO) across three load levels (45 experiments total), we reveal a fundamental trade-off: A2C achieves peak performance (771,222 at load 5.0$\times$) but exhibits high training variance (CV=28.22\%), while HCA2C demonstrates exceptional stability (CV=0.22\%) but limited performance and catastrophic failure under extreme load. This analysis provides evidence-based guidance for algorithm selection based on application requirements.

\item \textbf{Practical Design Guidelines:} We provide actionable recommendations for UAM system operators, including capacity configuration strategies, algorithm selection criteria based on application requirements (safety-critical vs. performance-critical), and extreme load management approaches.
\end{enumerate}

\textbf{Significance and Novelty:}

This research addresses a critical gap in the literature by providing the first systematic comparison of DRL algorithms for vertical queueing systems in UAM contexts. Our findings advance both theoretical understanding of DRL applications in operations research and provide actionable insights for UAM system designers. The demonstrated robustness across heterogeneous traffic patterns and insensitivity to reward function specifications establishes confidence for practical deployment.

\textbf{Fit with Applied Soft Computing:}

This manuscript aligns perfectly with the scope of \textit{Applied Soft Computing}, which publishes research on soft computing techniques applied to real-world problems. Our work combines deep reinforcement learning (a core soft computing technique) with operations research applications in emerging urban air mobility systems, demonstrating both theoretical advances and practical applicability.

\textbf{Manuscript Details:}
\begin{itemize}
\item Length: 46 pages (within 20-50 page requirement)
\item Abstract: 250 words (within 200-250 word requirement)
\item Keywords: 7 (within 5-7 requirement)
\item Figures: 19 high-quality figures (300 DPI)
\item Tables: 9 comprehensive tables
\item References: Comprehensive citations covering DRL, queueing theory, and UAM
\item Word count: Approximately 17,000 words
\end{itemize}

\textbf{Originality and Ethics:}

We confirm that:
\begin{itemize}
\item This manuscript is original research not previously published
\item The manuscript is not under consideration elsewhere
\item All authors have approved the submission
\item All ethical guidelines have been followed
\item No conflicts of interest exist
\item All data and code will be made available upon publication
\end{itemize}

\textbf{Suggested Reviewers:}

We respectfully suggest the following experts as potential reviewers:

\begin{enumerate}
\item \textbf{Dr. [Name 1]}\\
[Position], [Institution]\\
Email: [email]\\
Expertise: Deep reinforcement learning, operations research

\item \textbf{Dr. [Name 2]}\\
[Position], [Institution]\\
Email: [email]\\
Expertise: Queueing theory, traffic management

\item \textbf{Dr. [Name 3]}\\
[Position], [Institution]\\
Email: [email]\\
Expertise: Urban air mobility, autonomous systems

\item \textbf{Dr. [Name 4]}\\
[Position], [Institution]\\
Email: [email]\\
Expertise: Machine learning applications, optimization

\item \textbf{Dr. [Name 5]}\\
[Position], [Institution]\\
Email: [email]\\
Expertise: Reinforcement learning, multi-agent systems
\end{enumerate}

We believe this manuscript will be of significant interest to the \textit{Applied Soft Computing} readership and will contribute valuable insights to the growing field of AI-driven urban air mobility systems. We look forward to your consideration and welcome any feedback.

\closing{Sincerely,}

\end{letter}

\end{document}
