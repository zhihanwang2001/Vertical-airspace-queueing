% ============================================================
% Batch 3: System Model and Problem Formulation
% ~2,300 words total
% Created: 2026-01-05
% Based on: Documentation/guides/Final_Paper_Chinese_Version.md §3
% ============================================================

\section{System Model and Problem Formulation}
\label{sec:system_model}

\subsection{Scenario Description}

We consider an urban drone delivery network operating in \textbf{low-altitude corridors} partitioned into $L=5$ discrete vertical layers at altitudes 20m, 40m, 60m, 80m, and 100m. The system serves multiple service classes (standard/priority/emergency) with temporally correlated arrivals requiring layer-aware routing and capacity-constrained batch service.

\textbf{System Parameters and Units.} We adopt discrete-time modeling with time step $\Delta t = 1$ second. Service rates $\mu_\ell$ are measured in \textbf{items per second}, corresponding to single-step completion probability $p_\ell = 1 - e^{-\mu_\ell \cdot \Delta t}$ under exponential service time assumptions. Base arrival rates for the three order classes are $\lambda_1 = 0.30$ s$^{-1}$ (standard), $\lambda_2 = 0.15$ s$^{-1}$ (priority), and $\lambda_3 = 0.05$ s$^{-1}$ (emergency), with total arrival rate $\lambda_{\text{total}} = \sum_k \lambda_k = 0.50$ s$^{-1}$. Actual arrival rate for class $k$ at layer $\ell$ is $\lambda_{k,\ell} = \alpha_\ell \cdot \lambda_k$, where $\alpha_\ell$ denotes layer allocation weights representing traffic distribution across altitudes.

\textbf{Physical Constraints and Operational Context.} The vertical stratification reflects real-world urban airspace constraints: lower altitudes face building density, tree canopy, and ground-level activity restrictions, limiting available airspace; higher altitudes offer spatial freedom but introduce coordination complexity and weather exposure. Traffic distribution follows observed patterns in urban logistics: higher layers (L4-L5: 80--100m) handle 55\% of orders leveraging cruise-phase efficiency near the nominal 120m flight altitude, while ground-proximate layers (L1-L2: 20--40m) serve 25\% for direct building access.

\subsection{Vertical Layered Airspace}

\textbf{Layer Structure.} Let $\ell \in \{1, 2, 3, 4, 5\}$ index layers from lowest to highest altitude. Each layer $\ell$ is characterized by:
\begin{itemize}
\item \textbf{Capacity} $C_\ell$: Maximum number of orders that can queue simultaneously
\item \textbf{Service rate} $\mu_\ell$: Rate at which orders are completed (items/second)
\item \textbf{Altitude band} $[h^{(\ell)}_{\min}, h^{(\ell)}_{\max}]$: Vertical extent of the layer
\end{itemize}

We adopt an \textbf{inverted pyramid capacity profile} (higher capacity at higher altitudes) as the baseline configuration, contrasted with normal pyramid in our structural comparison experiments (§6.3):

\begin{center}
\begin{tabular}{clll}
\toprule
Layer & Altitude & Capacity $C_\ell$ & Service Rate $\mu_\ell$ \\
\midrule
L5 & 100 m & 8 & 1.20 items/s \\
L4 & 80 m & 6 & 1.00 items/s \\
L3 & 60 m & 4 & 0.80 items/s \\
L2 & 40 m & 3 & 0.60 items/s \\
L1 & 20 m & 2 & 0.40 items/s \\
\bottomrule
\end{tabular}
\end{center}

\textbf{Physical Rationale for Configuration.} This design reflects real urban low-altitude airspace physics and operational strategies:

\begin{enumerate}
\item \textbf{Altitude-Dependent Capacity}: Near-ground layers (20--40m) face constrained airspace due to buildings, trees, and pedestrian activity, supporting smaller queues. Higher layers (80--100m) offer open space, accommodating larger buffers. This \textbf{inverted pyramid} (C$_{20\text{m}}$=2 $<$ C$_{100\text{m}}$=8) matches physical constraints.

\item \textbf{Cruise-Proximity Service Rates} ($\mu$ increases with altitude): Lightweight delivery drones operate optimally at ~120m cruise altitude, making higher layers more efficient. Layer 5 (100m) enjoys fast horizontal maneuvering close to the 120m cruise corridor ($\mu_5=1.2$ fastest), while Layer 1 (20m) requires vertical climb/descent overhead ($\mu_1=0.4$ slowest). This gradient reflects transition time penalties between delivery endpoints and cruise altitude.
\end{enumerate}

\textbf{Traffic Distribution Pattern.} Order arrivals follow altitude allocation weights $\alpha = [0.10, 0.15, 0.20, 0.25, 0.30]$ from L1 to L5, reflecting delivery logistics: 55\% of orders route through higher layers (L4-L5) leveraging cruise efficiency, 25\% target ground-proximate layers (L1-L2) for direct building access, and 20\% use mid-layer (L3). This creates the \textbf{structural matching challenge}: low-traffic L1 (10\% of arrivals) must operate with minimal capacity (C$_1$=2), while high-traffic L5 (30\% of arrivals) has ample capacity (C$_5$=8). Our experiments (§6.3) demonstrate that this traffic-capacity alignment (inverted pyramid: capacity increases with altitude) outperforms mismatched configurations (normal pyramid) by +124\% in reward.

\subsection{MCRPS/SD/K Network Formulation}

The acronym \textbf{MCRPS/SD/K} encodes three foundational modeling components addressing limitations of classical queueing theory:

\begin{itemize}
\item \textbf{MC}: \textbf{M}ulti-\textbf{C}lass orders with three service priorities (standard/priority/emergency) and temporally \textbf{C}orrelated arrivals violating independent Poisson assumptions
\item \textbf{RPS}: \textbf{R}efined \textbf{P}oisson \textbf{S}plitting mechanism for distributing correlated aggregate flows across vertical layers
\item \textbf{SD}: \textbf{S}tate-\textbf{D}ependent control adapting service rates and inter-layer transfers based on real-time queue congestion
\item \textbf{K}: Finite capacity constraints $\sum_k n^t_{k,\ell} \leq C_\ell$ with blocking/overflow dynamics
\end{itemize}

\textbf{State Variables.} At discrete time $t$, the system state is defined by:
\begin{align}
n^t_{k,\ell} &\,:\, \text{Number of class-}k\text{ orders queued at layer }\ell \label{eq:state_queue} \\
Q^t_\ell &= \sum_{k=1}^{3} n^t_{k,\ell} \,:\, \text{Total queue length at layer }\ell \label{eq:state_total} \\
U^t_\ell &= Q^t_\ell / C_\ell \,:\, \text{Occupancy rate (utilization)} \label{eq:state_occupancy}
\end{align}

\textbf{Correlated Multi-Class Arrivals.} Unlike classical M/M/c models assuming independent Poisson arrivals, urban delivery orders exhibit temporal correlation (meal delivery surges during lunch/dinner, weather-triggered clustering). We model aggregate arrivals $A^t_{\text{total}}$ as a compound process, then split to layers and classes. For layer $\ell$ and class $k$, the arrival count at time $t$ is:
\begin{equation}
A^t_{k,\ell} \sim \text{Poisson}(\alpha_\ell \cdot \lambda_k \cdot \Delta t)
\label{eq:arrival_split}
\end{equation}
where $\alpha_\ell$ weights satisfy $\sum_{\ell=1}^{5} \alpha_\ell = 1$. Temporal correlation is captured via time-varying base rates $\lambda_k(t)$ following surge patterns, though our high-load experiments (§6) use constant rates scaled by 10$\times$ multiplier to stress-test extreme conditions.

\textbf{Validation of Poisson Approximation.} Kolmogorov-Smirnov tests on 10,000-step simulation traces confirm layer-wise inter-arrival intervals closely follow exponential distributions (typical $p$-values 0.3--0.8), validating conditional Poisson splitting for moderate correlation (correlation matrix elements 0.1--0.4). The approximation deteriorates under strong correlation (elements $\geq$0.7), where multivariate Poisson or Cox process modeling would be required---deferred to future work.

\textbf{Stochastic Batch Service.} Each layer's service capacity $\text{service\_capacity}_\ell = \min(C_\ell, Q^t_\ell)$ determines the maximum orders serviceable per time step. Theoretically, completed orders should follow \textbf{binomial batch service}:
\begin{equation}
S^t_\ell \sim \text{Binomial}(\text{service\_capacity}_\ell,\, p_\ell)
\quad \text{where} \quad p_\ell = 1 - e^{-\mu_\ell \Delta t}
\label{eq:service_binomial}
\end{equation}

\textbf{Current Implementation Note.} For computational efficiency, our experiments use a \textbf{Poisson approximation} $S^t_\ell \sim \text{Poisson}(\mu_\ell) + 1$ (+1 to avoid zero-service), which provides reasonable stochastic modeling for $\mu_\ell \in [0.4, 1.2]$ with stable performance metrics (§6). Full binomial batch service implementation (algorithm in Appendix B.2) is reserved for future work requiring precise service batch distribution modeling. Within-batch processing follows FCFS for same-class orders, with cross-class prioritization (emergency $>$ priority $>$ standard).

\textbf{Pressure-Triggered Downward Transfers.} Orders within queues can transfer between layers based on \textbf{pressure differentials} and fairness constraints, primarily implementing \textbf{downward transfers} (higher to lower layers) to balance loads during congestion surges. Transfer decisions use probabilistic mechanisms driven by \textbf{pressure gradients} $\Delta P^t_\ell$ and capacity states (detailed in §4.1).

\textbf{Design Rationale for Downward Transfers.} Although higher layers (L5: $C=8$, $\mu=1.2$) offer faster service, downward transfers to lower layers serve as \textbf{temporary overflow buffers} preventing global blockage during extreme peaks:
\begin{itemize}
\item \textbf{Normal operation}: System prioritizes high-layer efficiency via dynamic service rate modulation, rarely triggering transfers (experimental frequency $<$0.2\%)
\item \textbf{Peak shaving}: During traffic bursts, L5$\rightarrow$L1 transfers reduce peak queue lengths by 37\%, providing emergency relief
\item \textbf{Physical plausibility}: Lower layers, though slower ($\mu_1=0.4$), serve as short-term spillover buffers superior to rejecting orders; backlog gradually clears via priority mechanisms after high-layer pressure subsides
\end{itemize}

\textbf{Stability Sufficient Conditions.} Under arrival control soft-constraining each layer to $\rho_\ell < 1$ (where $\rho_\ell = \lambda_\ell / (\mu_\ell C_\ell)$ is traffic intensity), downward transfers diluting high-layer congestion guarantee \textbf{global queue positive recurrence} (Harris recurrence) and \textbf{fluid limit convergence} (Kurtz theorem). Detailed drift analysis appears in Appendix C.

\textbf{State-Dependent Control.} Control policies depend on \textbf{global queue state} $\mathbf{s}_t = \{n^t_{k,\ell}\}$ (§4.2) and enforce \textbf{finite capacity} constraints $\sum_k n^t_{k,\ell} \leq C_\ell$ with blocking for overflow attempts.

\textbf{Capacity Constraint Enforcement.} When $Q^t_\ell = C_\ell$ (layer full), new arrivals are either:
\begin{itemize}
\item \textbf{Blocked}: Rejected with penalty (crash event in evaluation metrics)
\item \textbf{Redirected}: Routed to adjacent layer with available capacity
\end{itemize}
Our DRL policies (§4.3) learn to balance service rate allocation and preventive transfers to minimize blocking probability, with varying robustness across algorithms under high-load conditions.

\subsection{Multi-Objective Problem Statement}

We formulate the control problem as a \textbf{multi-objective Markov decision process} (MOMDP) optimizing five core objectives plus one diagnostic metric. Objectives J$_1$--J$_5$ drive training, while J$_6$ (transfer efficiency) is evaluated post-hoc for diagnostic purposes.

\textbf{Six Evaluation Objectives (Fixed Definitions).} All objectives are measured over evaluation episodes of length $T=200$ steps for on-policy algorithms (A2C, PPO) and $T=10{,}000$ steps for off-policy algorithms (TD7):

\begin{enumerate}
\item \textbf{J$_1$ Throughput} ($\uparrow$): Mean completed orders per step, items/step. Higher is better.
\begin{equation}
J_1(\pi) = \frac{1}{T} \sum_{t=1}^{T} \sum_{\ell=1}^{5} S^t_\ell
\end{equation}

\item \textbf{J$_2$ Average Delay} ($\downarrow$): Mean waiting time from arrival to completion (time steps). Lower is better. Let $N_c$ = number of completed orders, $t_c$ = completion time, $t_a$ = arrival time.
\begin{equation}
J_2(\pi) = \frac{1}{N_c} \sum_{i=1}^{N_c} (t_{c,i} - t_{a,i})
\end{equation}

\item \textbf{J$_3$ Fairness (1$-$G)} ($\uparrow$): Load balance across layers via Gini coefficient complement. Higher is better (more fair).
\begin{equation}
J_3(\pi) = 1 - G_t \quad \text{where} \quad G_t = \frac{\sum_{i=1}^{5} \sum_{j=1}^{5} |U_i - U_j|}{2 \cdot 5^2 \cdot \bar{U}}
\end{equation}
with $\bar{U} = \frac{1}{5}\sum_{\ell=1}^{5} U_\ell$ the mean occupancy rate.

\item \textbf{J$_4$ Stability} ($\uparrow$): Inverse of queue length standard deviation (normalized). Higher is better (more stable).
\begin{equation}
J_4(\pi) = 1 - \frac{\sigma_Q}{\sigma_{\max}} \quad \text{where} \quad \sigma_Q = \sqrt{\frac{1}{T} \sum_{t=1}^{T} (Q^t_{\text{total}} - \bar{Q})^2}
\end{equation}

\item \textbf{J$_5$ Safety} ($\uparrow$): Complement of violation event rate (overflow/blocking). Higher is better (safer).
\begin{equation}
J_5(\pi) = 1 - \frac{N_{\text{violations}}}{T}
\end{equation}

\item \textbf{J$_6$ Transfer Efficiency}: Ratio of inter-layer transfers to total orders, \textbf{diagnostic only} (not optimized). Ideal value $<$0.2\% indicates minimal reliance on emergency transfers.
\begin{equation}
J_6(\pi) = \frac{\sum_{t=1}^{T} \sum_{\ell=1}^{4} \mathbb{1}[\text{transfer}_{\ell \to \ell-1}^t > 0]}{\sum_{t=1}^{T} A^t_{\text{total}}}
\end{equation}
\end{enumerate}

\textbf{Training Objective Mapping.} Training optimizes J$_1$--J$_5$ via weighted sum (details §4.3), where J$_2$ (delay) is approximated by \textbf{congestion penalty} (queue overflow instant feedback) and J$_4$ (stability) by instability penalty. J$_5$ (safety) is implicitly enforced via congestion penalties (overflow = violation). J$_6$ is computed during evaluation as a diagnostic indicator of transfer mechanism dependency. The composite training objective is:
\begin{equation}
\max_\pi \, J(\pi) = \sum_{i=1}^{5} w_i J_i(\pi)
\label{eq:multiobjective}
\end{equation}
subject to capacity constraints $\sum_k n^t_{k,\ell} \leq C_\ell$, flow conservation, and feasibility constraints. Training reward weight design appears in §4.3.

\textbf{Pareto Optimality.} A policy $\pi^*$ is \textbf{Pareto-optimal} if no alternative policy $\pi'$ exists such that $J_i(\pi') \geq J_i(\pi^*)$ for all $i \in \{1,\ldots,5\}$ with strict inequality for at least one objective. We identify the Pareto frontier via systematic sampling across algorithm hyperparameters, weight configurations, random seeds, and threshold settings (§4.2), yielding 262 Pareto-optimal solutions from 10,000 evaluated samples (2.62\%). Multi-criteria decision making (MCDM) extracts 13 representative knee-points for decision support.

% ============================================================
% Data authenticity statement
% ============================================================
% Capacity and service rate values: Final_Paper_Chinese_Version.md §3.2
% Traffic allocation weights: Chinese guide §3.1
% Objective definitions: Chinese guide §3.4
% Physical rationale: Chinese guide §3.2 physical constraints section
% ============================================================
